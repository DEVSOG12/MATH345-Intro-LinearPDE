\documentclass[]{article}

\usepackage{fullpage, amssymb, amsmath, mathrsfs, graphicx,setspace,bm}



\begin{document}
\begin{flushright} Oreofe Solarin \end{flushright}
\noindent MATH 345 Homework 5 Solutions  \\

\noindent \textbf{Question 1}

In SL-form:
\[
- \frac{d}{dx} \Bigl[(1+x^2)\phi'(x) \Bigr] = \lambda\phi(x) 
\]

and we have \(\phi(0) = 0\) and \(\phi(1) = 0\)


\noindent The Rayleigh quotient for a given \(u\) is:
\[
R[u] = \frac{\displaystyle\int_{0}^{1}p(x)[u'(x)]^2dx}{\displaystyle\int_{0}^{1}[u(x)]^2dx} = \frac{\displaystyle\int_{0}^{1}(1+x^2)[u'(x)]^2dx}{\displaystyle\int_{0}^{1}u(x)^2dx}
\]

\noindent for such \(u\) that satisfies the boundary conditions, we have \(\lambda_1 \leq R[u]\) \\

\noindent For our test function \(u_T(x)\), the first eigenfunction should be positive on \((0,1)\), satisfy \(\phi(0)=0, \phi(1)=0\) and have \(0\) derivative at \(x=0\) \\

\noindent since \(u_T(0) =0,\, u'_T(0) =1-0=1,\, u_T(1)=0\)


\[
u_T(x) = x(1-x)
\]

\[
[u_T(x)]^2 = [x(1-x)]^2 = x^2(1-x)^2 = x^2(1-2x+x^2) = x^2-2x^3+x^4
\]


\[
u'_T(x) = x\cdot-1 + (1-x)\cdot 1 = 1-2x
\]

\[
[u'_T(x)]^2 = [1-2x]^2 = 1 -2x-2x+4x^2 = 1-4x+4x^2
\]

Numerator: 

\begin{align*}
N &= \displaystyle\int_{0}^{1}(1+x^2)[u'_T(x)]^2 dx = \displaystyle\int_{0}^{1} (1+x^2)(1-4x+4x^2)dx \\
    &= \displaystyle\int_{0}^{1} 1-4x+4x^2+x^2-4x^3+4x^4dx = \displaystyle\int_{0}^{1} 1-4x+5x^2-4x^3+4x^4dx \\
    &= \Big[x-2x^2+\frac{5}{3}x^3-x^4+\frac{4}{5}x^5\Big]_{0}^{1} \\
    &= 1-2+\frac{5}{3}-1+\frac{4}{5}=\frac{7}{15}
\end{align*}

% \noindent using \(\displaystyle\int_{0}^{1}x^n dx = \frac{1}{n+1}\) \\

% \noindent we have:
% \[
% N = 4 \Bigl(\frac{1}{3}+ \frac{1}{5}\Bigr) = 4 \cdot \frac{8}{15} = \frac{32}{15}
% \]

Denominator: 
\begin{align*}
    D &= \displaystyle\int_{0}^{1} u_T(x)^2dx = \displaystyle\int_{0}^{1}x^2-2x^3+x^4dx \\
    &= \Big[\frac{1}{3}x^3-\frac{1}{2}x^4+\frac{1}{5}x^5\Big]^{1}_{0} = \frac{1}{3} - \frac{1}{2} + \frac{1}{5} = \frac{1}{30}
    % &= 1 - 2 \cdot \frac{1}{3} + \frac{1}{5} = 1 - \frac{2}{3} + \frac{1}{5} = \frac{8}{15}
\end{align*}

\noindent Our Rayleigh Q: 
\[
R[u_T] = \frac{N}{D} = \frac{7/15}{1/30} = 14
\]

\noindent So reasonable upper bound for the smallest eigenvalue is \(\boxed{\lambda_1 \leq 14}\) If we use higher order terms in our \(u_T\) function we can get more accurate estitmates

\pagebreak

\noindent \textbf{Question 2}

We have: 
\[
\frac{d}{dx}\Bigl(p(x)\frac{d\phi}{dx}\Bigr) + [\lambda\sigma(x) +q(x)]\phi = 0
\]

\noindent with BCs \(\frac{d\phi}{dx}(0) = 0\) and \(\frac{d\phi}{dx}(L) = 0\)\\ 

\noindent for \(\lambda \gg 1\), \(\lambda\sigma(x)\) dominates so we can drop \(q(x)\). and we now have \((p\phi')' + \lambda\sigma\phi  \approx 0\)


\[
\frac{\phi''p}{p}  + \frac{p'\phi}{p}  + \frac{\lambda\sigma\phi}{p} \approx 0
\]
\[
\phi''  +  \frac{p'\phi}{p} + \frac{\lambda\sigma\phi}{p}  \approx 0
\]

\noindent using ansatz:
\[
\phi(x) = A(x)\cos{\theta(x)}
\]

\[
\phi' = A'\cos{\theta(x)} - A\theta'\sin{\theta(x)}
\]

\[
\phi'' = (A''-A(\theta')^2)\cos\theta -(2A'\theta' + A\theta'')\sin\theta
\]

\[
(A''-A(\theta')^2)\cos\theta
-(2A'\theta' + A\theta'')\sin\theta
+\frac{\lambda\sigma}{p}A\cos\theta \approx 0
\]

\noindent cos coefficient:
\[
  A\Big(-(\theta')^2 + \lambda\frac{\sigma}{p}\Big) + A''.
\]
\noindent sin coefficient:
\[
  -(2A'\theta' + A\theta'')
\]

\noindent For large \(\lambda\), the terms \(A(\theta')^2\) and \(\lambda\frac{\sigma}{p}A\) are \(O(\lambda)\), while \(A''\) is \(O(1)\); thus we can drop \(A''\) at leading order:

\[
-(\theta')^2 + \lambda\frac{\sigma(x)}{p(x)}\approx0
\quad\Rightarrow\quad
(\theta')^2 \approx \lambda\frac{\sigma(x)}{p(x)}
\]

\[
\theta'(x)\approx \sqrt{\lambda}\sqrt{\frac{\sigma(x)}{p(x)}}.
\]

\[
\theta(x)\approx \sqrt{\lambda}\int_0^x\sqrt{\frac{\sigma(s)}{p(s)}}ds + C
\]

\[
\phi(x)\approx A(x)\cos\Big(\sqrt{\lambda}\int_0^x\sqrt{\frac{\sigma}{p}}ds + C\Big)
\]

\noindent using the Neumann boundary conditions \\

\noindent we have \(\phi'(0)=\phi'(L)=0\)

\[
\phi'(x) \approx -A\theta'(x)\sin\theta(x)
\]

\noindent so \\

\noindent at \(x=0\):

\[
  \phi'(0)\approx -A\theta'(0)\sin\theta(0)=0
  \quad\Rightarrow\quad \sin\theta(0)\approx0.
\]

\noindent we can take \(\theta(0)=0\) (so \(\delta=0)\)

\noindent at \(x=L\):

\[
  \phi'(L)\approx -A\theta'(L)\sin\theta(L)=0
  \quad\Rightarrow\quad \sin\theta(L)\approx0
  \quad\Rightarrow\quad \theta(L)\approx n\pi,\ n=0,1,2,\dots
\]

\noindent but

\[
\theta(L) \approx \sqrt{\lambda}\int_0^L\sqrt{\frac{\sigma(s)}{p(s)}}ds
=\sqrt{\lambda}J.
\]

\noindent so the quantization condition is:
\[\sqrt{\lambda_n}\,J \approx n\pi
\quad\Rightarrow\quad
\boxed{\lambda_n \approx \frac{n^2\pi^2}{J^2}},\qquad
J=\int_0^L\sqrt{\frac{\sigma(x)}{p(x)}}\,dx
\]

\noindent and the corresponding eigenfunctions (to leading order) are

\[
\boxed{\phi_n(x)\approx A\cos\left(
\frac{n\pi}{J}\int_0^x\sqrt{\frac{\sigma(s)}{p(s)}}ds
\right)}
\]

\pagebreak

\noindent \textbf{Question 3}



\noindent WWS, for continuous \(f:[0,\pi]\to\mathbb{R}\) (or even \(\mathbb{C}\)),

\[
\sum_{k=1}^{n} \left|\int_{0}^{\pi} f(x)\sin(kx)\,dx\right|^{2}
\le \frac{\pi}{2}\int_{0}^{\pi}|f(x)|^{2}\,dx 
\]



\noindent using orthogonality of the sin

\noindent the inner product:
\[
\langle g,h\rangle = \int_{0}^{\pi} g(x)\overline{h(x)}\,dx.
\]

\noindent For \(s_k(x)=\sin(kx)\) we have

\noindent For \(k\ne m\):
  \[
  \int_0^\pi \sin(kx)\sin(mx)\,dx=0,
  \]
\noindent For \(k=m\):
  \[
  \int_0^\pi \sin^2(kx)\,dx=\frac{\pi}{2}
  \]

\noindent So \({s_1,\dots,s_n}\) is an orthogonal set and
\[
|s_k|^{2}=\langle s_k,s_k\rangle=\frac{\pi}{2}
\]

\noindent For any constants \(c_1,\dots,c_n\) we consider
\[
g(x)=\sum_{k=1}^{n} c_k s_k(x)
\]


\noindent and since \(|f-g|^{2}\ge0\),
\[
0\le |f-g|^{2}
= |f|^{2} - 2\Re\sum_{k=1}^{n} c_k \langle f,s_k\rangle
+\sum_{k=1}^{n} |c_k|^{2} |s_k|^{2}
\]

\noindent using \(|s_k|^{2}=\pi/2\), this is
\[
0\le
|f|^{2} - 2\Re\sum_{k=1}^{n} c_k \langle f,s_k\rangle
+ \frac{\pi}{2}\sum_{k=1}^{n}|c_k|^{2} \tag{*}
\]

\noindent we choose
\[
c_k = \frac{2}{\pi}\langle f,s_k\rangle
= \frac{2}{\pi}\int_{0}^{\pi} f(x)\sin(kx)\,dx
\]

\noindent Plug into (*):

\noindent Middle term:
\[
-2\Re\sum c_k\langle f,s_k\rangle
= -2\sum \frac{2}{\pi}|\langle f,s_k\rangle|^{2}
= -\frac{4}{\pi}\sum_{k=1}^{n}|\langle f,s_k\rangle|^{2}
\]

\noindent Last term:
\[
\frac{\pi}{2}\sum |c_k|^{2}
=\frac{\pi}{2}\sum \left(\frac{2}{\pi}\right)^{2}|\langle f,s_k\rangle|^{2}
=\frac{2}{\pi}\sum_{k=1}^{n}|\langle f,s_k\rangle|^{2}
\]

\noindent so
\[
0\le |f|^{2}
-\frac{4}{\pi}\sum|\langle f,s_k\rangle|^{2}
+\frac{2}{\pi}\sum|\langle f,s_k\rangle|^{2}
= |f|^{2}
-\frac{2}{\pi}\sum_{k=1}^{n}|\langle f,s_k\rangle|^{2}
\]

\noindent rearranging:
\[
\sum_{k=1}^{n} |\langle f,s_k\rangle|^{2}
\le \frac{\pi}{2}|f|^{2}
\]

Finally, \(\langle f,s_k\rangle = \displaystyle\int_0^\pi f(x)\sin(kx)\,dx\) and
\(|f|^{2} = \displaystyle\int_0^\pi |f(x)|^{2}dx\), so this is exactly

\[
\boxed{
\sum_{k=1}^{n}\left|\int_{0}^{\pi} f(x)\sin(kx)\,dx\right|^{2}
\le \frac{\pi}{2}\int_{0}^{\pi}|f(x)|^{2}\,dx
}
\]

\pagebreak

\noindent \textbf{Question 4}
\[
\frac{\partial u}{\partial t}=k_1\frac{\partial^2u}{\partial x^2}
+k_2\frac{\partial^2u}{\partial y^2},\qquad
0<x<L;0<y<H,
\]
\noindent with
\(u(0,y,t)=u(L,y,t)=0\) (Dirichlet in (x)),\\
\(u_y(x,0,t)=u_y(x,H,t)=0\) (Neumann in (y)),\\
and \(u(x,y,0)=\alpha(x,y)\) \\

\noindent SoV:
\[
u(x,y,t)=X(x)Y(y)T(t)
\]
\[
X Y T' = k_1 X'' Y T + k_2 X Y'' T
\]
\[
\frac{T'}{T}=k_1\frac{X''}{X}+k_2\frac{Y''}{Y}
\]
\[
\frac{T'}{T}=-\lambda,\qquad k_1\frac{X''}{X}+k_2\frac{Y''}{Y}=-\lambda.
\]
\[
k_1\frac{X''}{X}+k_2\frac{Y''}{Y}=-\lambda
\quad\Rightarrow\quad
\frac{X''}{X}+\frac{k_2}{k_1}\frac{Y''}{Y}=-\frac{\lambda}{k_1}.
\]
\noindent separating again with another constant \(-\mu\):
\[
\frac{X''}{X}=-\mu,\qquad \frac{k_2}{k_1}\frac{Y''}{Y}=-(\tfrac{\lambda}{k_1}-\mu)
\]

\noindent ODEs:
\[
X''+\mu X=0,\qquad
Y''+\nu Y=0,\qquad
T'+\lambda T=0,
\]

\noindent where: \(\nu =\frac{k_1}{k_2}\left(\frac{\lambda}{k_1}-\mu\right)\) \\

\noindent For x: \(X''+\mu X=0,\ X(0)=X(L)=0\)
\[
\mu_n=\left(\frac{n\pi}{L}\right)^2,\qquad
X_n(x)=\sin\frac{n\pi x}{L},\qquad n=1,2,\dots
\]

\noindent For y: \(Y''+\nu Y=0,\ Y'(0)=Y'(H)=0\) \\

\noindent We have a neumann eigenproblem:
\[
\nu_m=\left(\frac{m\pi}{H}\right)^2,\qquad
Y_m(y)=\cos\frac{m\pi y}{H},\qquad m=0,1,2,\dots
\]


\noindent For each pair ((n,m)),

\[
\lambda_{nm}=k_1\mu_n+k_2\nu_m
= k_1\Big(\frac{n\pi}{L}\Big)^2+k_2\Big(\frac{m\pi}{H}\Big)^2
\]

\[
T_{nm}(t)=e^{-\lambda_{nm}t}
\]

\[
u(x,y,t)=\sum_{n=1}^{\infty}\sum_{m=0}^{\infty}
A_{nm}\times e^{-\left(k_1(n\pi/L)^2+k_2(m\pi/H)^2\right)t}
\sin\frac{n\pi x}{L}\times\cos\frac{m\pi y}{H}
\]

\noindent At \(t=0\),

\[
\alpha(x,y)=u(x,y,0)
=\sum_{n=1}^{\infty}\sum_{m=0}^{\infty}
A_{nm}\sin\frac{n\pi x}{L}\cos\frac{m\pi y}{H}
\]

\noindent so \(\alpha\) is expanded in the mixed sin-cos part

\noindent using orthogonality:

\(\displaystyle\int_0^L \sin\frac{n\pi x}{L}\sin\frac{p\pi x}{L}dx
  =\frac{L}{2}\delta_{np}\),
\(\displaystyle\int_0^H \cos\frac{m\pi y}{H}\cos\frac{q\pi y}{H}dy
  =\begin{cases}
  H, & m=q=0,\\[2pt]
  \frac{H}{2}\delta_{mq}, & m,q\ge1.
  \end{cases}\)

\noindent we get \\

\noindent for \(m\ge1\):
\[
  A_{nm}
  =\frac{4}{LH}\int_0^L\int_0^H
  \alpha(x,y)\sin\frac{n\pi x}{L}\cos\frac{m\pi y}{H}dydx;
\]

\noindent for \(m=0\):
\[
  A_{n0}
  =\frac{2}{LH}\int_0^L\int_0^H
  \alpha(x,y)\sin\frac{n\pi x}{L}dydx.
\]

\[
\boxed{
u(x,y,t)=\sum_{n=1}^{\infty}\sum_{m=0}^{\infty}
A_{nm}e^{-\left(k_1\left(\frac{n\pi}{L}\right)^2
+k_2\left(\frac{m\pi}{H}\right)^2\right)t}
\sin\frac{n\pi x}{L}\cos\frac{m\pi y}{H}
}
\]

\noindent with \(A_{nm}\) determined from \(\alpha(x,y)\) by the formulas above


\pagebreak

\textbf{Question 5}
\[
c(x,y,z)\rho(x,y,z) \frac{\partial u}{dt} =  \nabla \cdot (K_0 (x,y,z) \nabla u)
\]
% \[
% c(x,y,z)p(x,y,z) \frac{\partial u}{dt} =  \nabla \cdot (K_0 (x,y,z) (\frac{\partial u}{\partial x} + \frac{\partial u}{\partial y}+\frac{\partial  u }{\partial z}))
% \]
% \[
% cp \frac{\partial u }{dt} = \Delta u\cdot K_0
% \]
% \[
% cp \frac{\partial u }{dt} = k(\frac{\partial^2u}{\partial x} + \frac{\partial^2u}{\partial y}+\frac{\partial^2 u }{\partial z})
% \]

\[
u(x,y,z,t) = \phi(x,y,z)h(t)
\]
\[
\phi h' = \frac{1}{c\rho} (\nabla \cdot (K_0(h(t)\nabla\phi))) = \frac{h}{c\rho} \nabla \cdot (K_0\nabla\phi)
\]
\noindent since \(K_0\) depends only on space (last part)

\[
\frac{h'}{h} = \frac{1}{c\rho\phi}\nabla\cdot(K_o\nabla\phi)
\]
\noindent equate to \(-\lambda\)
\[
\frac{h'}{h} = \frac{1}{c\rho\phi}\nabla\cdot(K_o\nabla\phi) = -\lambda
\]
\[
\frac{h'}{h} = -\lambda \qquad \frac{1}{c\rho\phi}\nabla\cdot(K_o\nabla\phi) = -\lambda
\]
\noindent from the first \(h\): 
\[
h(t) = Ce^{-\lambda t}
\]

\[
\nabla\cdot(K_0\nabla\phi) = -\lambda c\rho\phi \iff \nabla\cdot(K_0\nabla\phi) + \lambda c\rho\phi = 0
\]

\noindent because we have \(u=0\) at the bds for all \(t\) and \(u=\phi h\), we must have \(\phi=0\) on the region \(\partial\Omega\) \\


\noindent we just found the eigenvalue eq.: 
\[
\nabla\cdot(K_0\nabla\phi) + \lambda c\rho\phi = 0
\]

\noindent and it matches expected
\[
\nabla\cdot(p\nabla\phi) + \lambda\sigma(x,y,z)\phi = 0
\]

\boxed{
    p(x,y,z) = K_0(x,y,z) \qquad \sigma(x,y,z) = c(x,y,z)\rho(x,y,z)
}


\end{document}