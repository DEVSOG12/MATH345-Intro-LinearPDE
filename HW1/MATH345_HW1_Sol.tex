\documentclass{article}
\usepackage{fullpage, amssymb, amsmath, mathrsfs, graphicx,setspace,bm}



\begin{document}
\begin{flushright} Oreofe Solarin \end{flushright}
\noindent MATH 345 Homework 1 Solutions  \\

\noindent \textbf{Prelimary Def:} Thermal Energy (T.E.), Heat Energy (H.E.), Heat Flux (H.F.), Thermal Energy Density (T.E.D.), Heat Source (H.S.), Specific Heat (S.H.)\\ 
%%%%%%%%%%%%%%%%%%%%%%%%%%%%%%%
% Problem 1.2.4
%%%%%%%%%%%%%%%%%%%%%%%%%%%%%%%

\noindent \textbf{Problem 1.2.4}
\begin{figure}[!h]
    \centering
    \includegraphics[width=0.9\linewidth]{figures/p124.jpeg}
    \caption{Rod}
\end{figure}

\noindent We define our surface area of the rod as \(A(x)\). 

\noindent T.E.D.: 
\[
 e(x,t)
\]

\noindent H.E.:
\[
 e(x,t) A(x) \cdot (b-a)
\]
\noindent where the volume of a slice is \(A(x) \cdot (b-a)\) and we define \((b-a)=\Delta x\).

\[
\frac{\partial}{\partial t} \Bigl[e(x,t) \cdot A(x) \Delta x \Bigr] \approx \phi(x,t)A(x) - \phi(x+\Delta x,t)A(x+\Delta x)
\]
\noindent (L: time rate of change of energy in slice. R: inflow at \(x\) minus outflow at \(x+\Delta x\) plus sources.)

\medskip

\noindent In the limit as \(\Delta x \to 0\), divide by \(\Delta x\) and we use
\[
\phi(x+\Delta x,t)A(x+\Delta x) = \phi(x,t)A(x) + \partial_x[\phi(x,t)A(x)]\,\Delta x + o(\Delta x)
\]
which yields
\[
\frac{\partial}{\partial t} [e(x,t) A(x)] = - \, \frac{\partial}{\partial x} [\phi(x,t)A(x)]
\]

\[
e(x,t)=\rho c\,T(x,t), \qquad \phi(x,t) = -\,k\,\frac{\partial T}{\partial x}(x,t).
\]

\[
\frac{\partial}{\partial t}\Bigl[\rho c\,A(x)\,T(x,t)\Bigr]
= -\,\frac{\partial}{\partial x}\Bigl[-\,k\,A(x)\,T_x(x,t)\Bigr].
\]
\noindent Since \(A=A(x)\) is time-independent and \(k,\rho,c\) are constants,
\[
\rho c\,A(x)\,T_t(x,t) \;=\; \frac{\partial}{\partial x}\Bigl(k\,A(x)\,T_x(x,t)\Bigr).
\]
\noindent Hence the heat eq. for a rod with \(A(x)\) and \(Q=0\):
\[
\,(\rho c)\,A(x)\,T_t \;=\; \bigl(k\,A(x)\,T_x\bigr)_x\,
\quad\Longleftrightarrow\quad
\,T_t \;=\; \alpha\,\frac{1}{A(x)}\,\frac{\partial}{\partial x}\Bigl(A(x)\,T_x\Bigr),\;\;\alpha=\frac{k}{\rho c}.
\]

\noindent One way we can confirm we have the right answer is to check if \(A(x)\equiv A_0\), then \((kA_0 T_x)_x = kA_0 T_{xx}\) and we get \(T_t=\alpha\,T_{xx}\).
\vspace{0.2\linewidth}


%%%%%%%%%%%%%%%%%%%%%%%%%%%%%%%
% Problem 1.2.7
%%%%%%%%%%%%%%%%%%%%%%%%%%%%%%%

\noindent \textbf{Problem 1.2.7}

\noindent Suppose that S.H. is \(c=c(x,u)\)

\noindent (a)

\noindent For a small temperature to raise $u\to u+du$ at fixed $x$ requires per unit mass heat

\[
dq = c(x,u)\,du.
\]

\[
q(x,t)=\int_{0}^{u(x,t)} c\bigl(x,\bar u\bigr)\,d\bar u,
\]

\medskip

\noindent (b)

\noindent T.E. per unit volume is
\[
e(x,t)=\rho(x)\,q(x,t)=\rho(x)\int_{0}^{u(x,t)} c\bigl(x,\bar u\bigr)\,d\bar u.
\]

\noindent Conservation of thermal energy

\[
\frac{\partial e}{\partial t} \;=\; -\,\frac{\partial \phi}{\partial x} \;+\; Q,
\]
\smallskip
\noindent\emph{Leibniz rule} Since $\bar u$ has no explicit $t$–dependence,
\[
\frac{\partial}{\partial t}\!\left[\int_{0}^{u(x,t)} c\bigl(x,\bar u\bigr)\,d\bar u\right]
= c\bigl(x,u(x,t)\bigr)\,u_t(x,t)
\]

\noindent Hence
\[
\frac{\partial e}{\partial t}
= \rho(x)\,c\bigl(x,u\bigr)\,u_t(x,t)
\]

\[
\rho(x)\,c\bigl(x,u\bigr)\,u_t
\;=\; -\,\phi_x \;+\; Q
\]
With Fourier’s law
\[
\phi(x,t) = -\,K_0(x)\,u_x \quad\Longrightarrow\quad
-\phi_x = \partial_x\!\bigl(K_0(x)\,u_x\bigr)
\]
\[
\;\rho(x)\,c\bigl(x,u\bigr)\,u_t
= \frac{\partial}{\partial x}\!\Bigl(K_0(x)\,u_x\Bigr) + Q(x,t)
\]
If $K_0,\rho$ are constants and $Q=0$ this is
\[
(\rho\,c(x,u))\,u_t = K_0\,u_{xx},
\qquad\quad
u_t = \frac{K_0}{\rho}\,\frac{u_{xx}}{c(x,u)}
\]


%%%%%%%%%%%%%%%%%%%%%%%%%%%%%%%
% Problem 1.4.2
%%%%%%%%%%%%%%%%%%%%%%%%%%%%%%%

\noindent \textbf{Problem 1.4.2}

\noindent Equilibrium $\Rightarrow u_t=0$. With constant properties and Fourier’s law
$\phi=-K_0 u_x$
\[
\frac{\partial e}{\partial t}=0=-\phi_x+Q \;\;\Longrightarrow\;\; (K_0 u_x)_x+Q=0
\quad\Rightarrow\quad K_0\,u_{xx}+Q=0
\]


\noindent (a)

% \begin{enumerate}
$Q=0$, $\;u_x(0)=0,\;u(L)=T$.
\[
u_{xx}=0 \;\Rightarrow\; u(x)=ax+b,\quad u_x=a.
\]
From $u_x(0)=0$ we get $a=0$, hence $u(x)\equiv b$. Using $u(L)=T$ gives
\[
\,u(x)\equiv T\,
\]

\noindent (b)

$\displaystyle \frac{Q}{K_0}=1\;\Rightarrow\;Q=K_0$, $\;u(0)=T_1,\;u(L)=T_2$.
\[
K_0 u_{xx}+K_0=0 \;\Rightarrow\; u_{xx}=-1.
\]
\[
u_x(x) = \int u_{xx}(x) \,dx = \int (-1) \,dx = -x + C_1
\]
\[
u(x) = \int u_x(x) \,dx = \int (-x + C_1) \,dx = -\frac{x^2}{2} + C_1 x + C_2
\]
Apply BCs:
\[
u(0)=T_1 \Rightarrow C_2=T_1,\qquad
u(L)=T_2 \Rightarrow -\frac{L^2}{2}+C_1 L+T_1=T_2 \Rightarrow
C_1=\frac{T_2-T_1}{L}+\frac{L}{2}
\]
Therefore
\[
u(x)=-\frac{x^2}{2}+\Bigl(\frac{T_2-T_1}{L}+\frac{L}{2}\Bigr)x+T_1
\;=\;
T_1+\frac{T_2-T_1}{L}\,x+\frac{x(L-x)}{2}
\]
\[
\,u(x)=T_1+\frac{T_2-T_1}{L}\,x+\frac{x(L-x)}{2}\,
\]

\pagebreak


%%%%%%%%%%%%%%%%%%%%%%%%%%%%%%%
% Problem 1.4.8
%%%%%%%%%%%%%%%%%%%%%%%%%%%%%%%

\noindent \textbf{Problem 1.4.8}
\[
u_{xx}=0 \quad\Longrightarrow\quad u(x)=a x + b,\; u_x(x)\equiv a
\]
BCs give
\[
u_x(0,t)=1 \Rightarrow a=1,\qquad u_x(L,t)=\beta \Rightarrow a=\beta
\]
Consistency $\Rightarrow \beta=1$

\medskip

\noindent If $\beta=1$, the family of steady states is
\[
\,u_{\mathrm{eq}}(x)=x+C,\;\;C\in\mathbb{R}\,
\]

\medskip
\noindent\emph{Physical explanation.}
Integrate the PDE over $(0,L)$:
\[
\frac{d}{dt}\int_0^L u\,dx=\int_0^L u_{xx}\,dx=u_x(L,t)-u_x(0,t)=\beta-1.
\]
Thus:
\[
\;\beta\neq 1\;\Rightarrow\;\text{net heat flux }\neq 0\;\Rightarrow\;\int_0^L u\,dx
\text{ drifts linearly in time }\Rightarrow\text{ no equilibrium.}\;
\]
\[
\;\beta=1\;\Rightarrow\;\text{net flux }=0\;\Rightarrow\text{ equilibrium exists.}\;
\]

\medskip
\noindent\emph{Which $C$?} When $\beta=1$, the spatial mean is conserved:
\[
\int_0^L u(x,t)\,dx=\int_0^L f(x)\,dx.
\]
\[
\int_0^L (x+C)\,dx=\int_0^L f(x)\,dx
\;\;\Rightarrow\;\;
C=\frac{1}{L}\int_0^L f(x)\,dx-\frac{L}{2}.
\]
So the steady state compatible with the initial data is
\[
\,u_{\mathrm{eq}}(x)=x+\Bigl(\frac{1}{L}\int_0^L f(x)\,dx-\frac{L}{2}\Bigr)\,
\quad\text{exists iff }\beta=1\text{.}
\]

\pagebreak


%%%%%%%%%%%%%%%%%%%%%%%%%%%%%%%
% Problem 1.4.11
%%%%%%%%%%%%%%%%%%%%%%%%%%%%%%%

\noindent \textbf{Problem 1.4.11}

\medskip
\noindent Let
\[
E(t)\;=\;\int_{0}^{L} u(x,t)\,dx
\]

\[
E'(t)\;=\;\int_{0}^{L} u_t\,dx
=\int_{0}^{L}\big(u_{xx}+4\big)\,dx
=\Big[u_x\Big]_{0}^{L}+4L
=u_x(L,t)-u_x(0,t)+4L
=6-5+4L
=1+4L.
\]

\[
\,E(t)=E(0)+(1+4L)\,t
\;=\;\int_{0}^{L} f(x)\,dx\;+\;(1+4L)\,t\,.
\]

\medskip
With constant density $\rho$, S.H. $c$, and cross-sectional area $A$, total T.E. is
\[
\,\mathcal{H}(t)=\rho\,c\,A\,E(t)
=\rho c A\!\left(\int_{0}^{L} f(x)\,dx+(1+4L)\,t\right).\,
\]

%%%%%%%%%%%%%%%%%%%%%%%%%%%%%%%
% Problem 1.5.3
%%%%%%%%%%%%%%%%%%%%%%%%%%%%%%%
\noindent\textbf{Problem 1.5.3.}\quad $x=r\cos\theta,\;y=r\sin\theta$.

\begin{enumerate}
\item[\textbf{(a)}] Since $r^2=x^2+y^2$,
\[
2r\,r_x=2x\Rightarrow r_x=\frac{x}{r}=\cos\theta,\qquad
2r\,r_y=2y\Rightarrow r_y=\frac{y}{r}=\sin\theta.
\]
Also $\theta=\arctan(y/x)$, so
\[
\theta_x=\frac{-y}{x^2+y^2}=-\frac{\sin\theta}{r},
\qquad
\theta_y=\frac{x}{x^2+y^2}=\frac{\cos\theta}{r}.
\] 

\item[\textbf{(b)}] Position vector $\vec r = x\,\hat{\vec i}+y\,\hat{\vec j}
= r(\cos\theta\,\hat{\vec i}+\sin\theta\,\hat{\vec j})$,
so
\[
\;\hat{\vec r}=\cos\theta\,\hat{\vec i}+\sin\theta\,\hat{\vec j}\;,\qquad
\;\hat{\bm\theta}=-\sin\theta\,\hat{\vec i}+\cos\theta\,\hat{\vec j}\;
\]
(Obtain $\hat{\bm\theta}$ by $+90^\circ$ rotation or by $\partial_\theta\hat{\vec r}$.) 

\item[\textbf{(c)}] Chain rule:
\[
\partial_x=r_x\,\partial_r+\theta_x\,\partial_\theta
=\cos\theta\,\partial_r-\frac{\sin\theta}{r}\,\partial_\theta,\quad
\partial_y=r_y\,\partial_r+\theta_y\,\partial_\theta
=\sin\theta\,\partial_r+\frac{\cos\theta}{r}\,\partial_\theta.
\]
Hence
\[
\nabla=\hat{\vec i}\,\partial_x+\hat{\vec j}\,\partial_y
=(\cos\theta\,\hat{\vec i}+\sin\theta\,\hat{\vec j})\,\partial_r
+(-\sin\theta\,\hat{\vec i}+\cos\theta\,\hat{\vec j})\,\frac{1}{r}\partial_\theta
=\boxed{\;\hat{\vec r}\,\partial_r+\hat{\bm\theta}\,\frac{1}{r}\partial_\theta\;}.
\]
Applying to $u$,
\[
\;\nabla u=\hat{\vec r}\,u_r+\hat{\bm\theta}\,\frac{1}{r}\,u_\theta\;
\]
\end{enumerate}

\end{document}