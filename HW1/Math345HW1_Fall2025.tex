\documentclass[12]{article}

\usepackage[pdftex]{graphicx}
\usepackage{subfig} 
\usepackage{    }
\usepackage{amsthm}
\usepackage{amsfonts}
\usepackage{multirow}
\usepackage{fullpage, amssymb, mathrsfs, graphicx,setspace,bm}
\usepackage[colorlinks=true,linkcolor=black,anchorcolor=black,citecolor=black,filecolor=black,menucolor=black,runcolor=black,urlcolor=blue]{hyperref}

\newcommand{\ds}{\displaystyle}
\newcommand{\oldst}{~|~}
\newcommand{\st}{:}
\newcommand{\lm}[2]{\ensuremath{\displaystyle\lim_{#1 \rightarrow #2}}}
\newcommand{\und}[1]{\underline{#1}}
\newcommand{\ub}[1]{\underline{\textbf{#1}}}

\renewcommand{\vec}[1]{\mathbf{#1}}



\begin{document}
%\begin{left}
\noindent {\Large Math 345} \hfill {\Large Fall 2025} \vspace{0.25cm}\\
{\large {\bf Homework \#1}} \vspace{0.1cm} \\
{\large {\bf Due: September 8, 2025, 11:59 p.m.}} \vspace{0.5cm} \\
%\end{left}
{\it Show work to justify your answers and write neatly (or type) for full credit. Your work must be uploaded onto 
Gradescope (accessed through Canvas).} 
%Your code must be submitted electronically to Canvas. 
%Graphs generated in Matlab must be saved as .eps, .jpeg, or pdf files.}

\begin{enumerate}
%%%%%%%%%%%%%%%%%%%%%%%%%%%%%%%
%
% Problem 1
%
%%%%%%%%%%%%%%%%%%%%%%%%%%%%%%%
\item Problem 1.2.4. Derive the heat equation for a rod assuming constant thermal properties with variable cross-sectional area $A(x)$ assuming no sources by considering the total thermal energy between $x=a$ and $x=b$.
%%%%%%%%%%%%%%%%%%%%%%%%%%%%%%%
%
% Problem 2
%
%%%%%%%%%%%%%%%%%%%%%%%%%%%%%%%
\item Problem 1.2.7. Suppose that the specific heat is a function of position and temperature, $c(x,u)$.
\begin{enumerate}
\item Show that the heat energy per unit mass necessary to raise the temperature of a thin slice of thickness $\Delta x$ from $0^{\circ}$ to $u(x,t)$ is not $c(x)u(x,t)$, but instead 
$\ds{\int_0^u c(x,\bar{u}) \, d \bar{u}}$. 
\item Rederive the heat equation in this case. Show that
\[
\frac{\partial e}{\partial t} = - \frac{\partial \phi}{\partial x} + Q
\]
remains unchanged. Hint: Assume $\bar{u}$ is not a function of time. Apply Leibniz’s rule.
\end{enumerate}
%%%%%%%%%%%%%%%%%%%%%%%%%%%%%%%
%
% Problem 3
%
%%%%%%%%%%%%%%%%%%%%%%%%%%%%%%%
\item Problem 1.4.2. Determine the equilibrium temperature distribution for a one-dimensional rod 
with constant thermal properties with the following sources and boundary conditions:
\begin{enumerate}
\item $Q = 0$, $\ds{\frac{\partial u}{\partial x} (0) = 0},$ $u(L) = T$.
\item $\ds{\frac{Q}{K_0} = 1}$, $u(0) = T_1$, $u(L) = T_2$.
\end{enumerate}
%%%%%%%%%%%%%%%%%%%%%%%%%%%%%%%
%
% Problem 4
%
%%%%%%%%%%%%%%%%%%%%%%%%%%%%%%%
\item Problem 1.4.8. Determine an equilibrium temperature distribution (if one exists). For what values of $\beta$ are there solutions? Explain physically.
\[
\frac{\partial u}{\partial t} = \frac{\partial^2 u}{\partial x^2}, \qquad u(x,0) = f(x), \qquad 
\frac{\partial u}{\partial x}(0,t) = 1, \qquad \frac{\partial u}{\partial x}(L,t) = \beta.
\]


\item Problem 1.4.11. Suppose $\ds{\frac{\partial u}{\partial t} = \frac{\partial^2 u}{\partial x^2} + 4}$, $u(x,0) = f(x)$,  $\ds{\frac{\partial u}{\partial x}(0,t) = 5}$, and 
$\ds{\frac{\partial u}{\partial x}(L,t) = 6}$.  Calculate the total thermal energy in the one-dimensional rod (as a function of time).
%%%%%%%%%%%%%%%%%%%%%%%%%%%%%%%
%
% Problem 5
%
%%%%%%%%%%%%%%%%%%%%%%%%%%%%%%%
\item Problem 1.5.3. Note that we will start this problem in class. Consider the polar coordinates $x = r \cos \theta$, $y = r \sin \theta$.
\begin{enumerate}
\item Since $r^2 = x^2 + y^2$, show that 
\[
\frac{\partial r}{\partial x} = \cos \theta, \quad
\frac{\partial r}{\partial y} = \sin \theta, \quad
\frac{\partial \theta}{\partial y} = \frac{\cos \theta}{r}, \quad 
\frac{\partial \theta}{\partial x} = - \frac{\sin \theta}{r}.
\]
\item Show that $\hat{\vec{r}} = \cos \theta \, \hat{\vec{i}} +  \sin \theta \, \hat{\vec{j}}$ and 
$\hat{\bm \theta} = - \sin \theta \, \hat{\vec{i}} +  \cos \theta \, \hat{\vec{j}}$. 
\item Using the chain rule, show that
\[
\nabla  = \hat{\vec{r}} \frac{\partial }{\partial r} + \hat{\bm \theta} \frac{1}{r}  \frac{\partial }{\partial \theta} \text{, and hence }
\nabla u  = \hat{\vec{r}} \frac{\partial u }{\partial r} + \hat{\bm \theta} \frac{1}{r}  \frac{\partial u }{\partial \theta}.
\]
\end{enumerate}

%\begin{center}
%\includegraphics[width=0.6\textwidth]{Screenshot.png}
%\end{center}
\end{enumerate}
\end{document}