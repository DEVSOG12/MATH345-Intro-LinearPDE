\documentclass{article}
\usepackage{fullpage, amssymb, amsmath, mathrsfs, graphicx,setspace,bm}



\begin{document}
\begin{flushright} Oreofe Solarin \end{flushright}
\noindent MATH 345 Homework 2 Solutions  \\

%%%%%%%%%%%%%%%%%%%%%%%%%%%%%%%
% Problem 1.5.8
%%%%%%%%%%%%%%%%%%%%%%%%%%%%%%%
\noindent \textbf{Problem 1.5.8}

\noindent We can use the divergence theorem with $A = \nabla u$. If $u$ satisfies Laplaces's equation in $\mathbb{R}^3$ 
\medskip

\noindent \(
\nabla^2u=\nabla \cdot (\nabla u) = 0
\) in $V$  \\
\medskip
\noindent for any volume $V$ bounded by the closed surface $S$, then:  \\
\medskip
\[
\iint\limits_{S} \nabla u \cdot \hat{n} dS = \iiint\limits_V \nabla \cdot (\nabla u) dV = \iiint\limits_V \nabla^2u dV = 0
\]

which proves the claim for every closed surface $S$

\medskip

\noindent In a steady conduction with constant conductivity $K$ and no internal sources, the heat flux is $q=-K\nabla u$.
\[
\iint\limits_S q \cdot \hat{n} dS = - K \iint\limits_S \nabla u \cdot \hat{n} dS = 0
\]

\noindent so the net heat flow throught any clsoed surface is 0. The heat leaving through some parts of $S$ is exactly balanced by 
heat entering elsewhere. Equivalently, there are no sources/sinks of heat inside $S$ at steady state.



%%%%%%%%%%%%%%%%%%%%%%%%%%%%%%%
% Problem 2.2.5
%%%%%%%%%%%%%%%%%%%%%%%%%%%%%%%
\noindent \textbf{Problem 2.2.5}


\noindent \textbf{(a)} \\ 
If $L(u_p) = f$ and $L(u_1)= L(u_2) = 0$, then for any constant $c_1, c_2$, 

\[
L(u_p + c_1u_1 + c_2u_2) = L(u_p) + c_1L(u_1) + c_2L(u_2) = f + 0 + 0 = f
\]

\noindent so $u = u_p + c_1u_1 + c_2u_2$ is another particular solution to $L(u) =f$

\noindent \textbf{(b)} \\
If $L(u)= f_1 + f_2$ and $u_{p,1}, u_{p,2}$ satisfy $L(u_{p,1}) = f_1$ and  $L(u_{p,2}) = f_2$, then
\[
L(u_{p,1} + u_{p,2}) = L(u_{p,1}) + L(u_{p,2}) = f_1 + f_2
\]

so a particular solution for $f_1 + f_2$ is $u_p = u_{p,1} + u_{p,2}$

%%%%%%%%%%%%%%%%%%%%%%%%%%%%%%%
% Problem 2.3.1
%%%%%%%%%%%%%%%%%%%%%%%%%%%%%%%
\noindent \textbf{Problem 2.3.1}

\noindent Let $u(r,t) = R(r)T(t)$

\[
u_t = \frac{k}{r^2}\frac{\partial}{\partial r}(r^2 u_r)
\]
\[
R(r)T'(t) = \frac{k}{r^2}(\frac{d}{dr}(r^2R'(r)))T(t)
\]
\[
\frac{1}{k}\frac{T'}{T}=\frac{1}{r^2R}\frac{d}{dr}(r^2R') = -\lambda 
\]
ODEs:
\[
T'(t) + k\lambda T(t) = 0
\]

\[
\frac{1}{r^2}\frac{d}{dr}(r^2R'(r)) + \lambda R(r) = 0  \iff r^2R'' + 2rR' + \lambda r^2R =0
\]

\pagebreak

%%%%%%%%%%%%%%%%%%%%%%%%%%%%%%%
% Problem 2.3.2
%%%%%%%%%%%%%%%%%%%%%%%%%%%%%%%
\noindent \textbf{Problem 2.3.2}

\[
y'' + 4y = 0 \implies y(x) = A \cos(2x) + B \sin(2x)
\]

\noindent \textbf{(a)}

\[
y(0) = 3 \implies A = 3, y(1) = 5 \implies 3\cos(2) + B\sin(2) = 5
\]
\noindent $\sin(2) \neq 0$
\[
B = \frac{5-3\cos2}{\sin2}
\]

\[
y(x) = 3 \cos(2x) + \frac{5-3\cos2}{\sin2}\sin(2x)
\]

\noindent \textbf{(b)}

\[
y(\pi) = 3\cos(2\pi) + B \sin(2\pi) =3 \neq 5
\]
\noindent thus, no solution

\noindent \textbf{(c)}

\[
y(0) = 3 \implies A = 3
\]

\[
y(\pi) = 3\cos(2\pi) + B\sin(2\pi) = 3 \text{ for any } B
\]

\[
y(x) = 3\cos(2x) + B\sin(2x), B \in \mathbb{R} 
\]
\noindent thus, infinitely many solutions

%%%%%%%%%%%%%%%%%%%%%%%%%%%%%%%
% Problem 2.3.7
%%%%%%%%%%%%%%%%%%%%%%%%%%%%%%%
\noindent \textbf{Problem 2.3.7}

\noindent We consider two cases.

\noindent Case 1: $n=m$.
\begin{align*}
I &= \int_{0}^{L} \sin^2\left(\frac{n\pi x}{L}\right)dx\\
&= \int_{0}^{L} \frac{1}{2}\left(1 - \cos\left(\frac{2n\pi x}{L}\right)\right) dx\\
&= \frac{1}{2}\left[x - \frac{L}{2n\pi}\sin\left(\frac{2n\pi x}{L}\right)\right]_{0}^{L} \\
&= \frac{1}{2}(L - 0) = \frac{L}{2}
\end{align*}

\noindent Case 2: $n \neq m$.
\begin{align*}
I &= \int_{0}^{L} \sin\left(\frac{n\pi x}{L}\right)\sin\left(\frac{m\pi x}{L}\right)dx\\
&= \frac{1}{2}\int_{0}^{L} \left[ \cos\left(\frac{(n-m)\pi x}{L}\right) - \cos\left(\frac{(n+m)\pi x}{L}\right) \right] dx\\
&= \frac{1}{2}\left[\frac{L}{(n-m)\pi}\sin\left(\frac{(n-m)\pi x}{L}\right) - \frac{L}{(n+m)\pi}\sin\left(\frac{(n+m)\pi x}{L}\right)  \right]_{0}^{L} \\
&= \frac{1}{2} \left( \frac{L}{(n-m)\pi}\sin((n-m)\pi) - \frac{L}{(n+m)\pi}\sin((n+m)\pi) \right) - 0 \\
&= 0
\end{align*}
since $\sin(k\pi)=0$ for any integer $k$.

\noindent Combining these results, we get the orthogonality condition:
\[
\int_{0}^{L} \sin\left(\frac{n\pi x}{L}\right)\sin\left(\frac{m\pi x}{L}\right)dx = 
\begin{cases}
L/2 & \text{if } n=m \\
0 & \text{if } n \neq m
\end{cases}
\]

%%%%%%%%%%%%%%%%%%%%%%%%%%%%%%%
% Problem 2.4.4
%%%%%%%%%%%%%%%%%%%%%%%%%%%%%%%
\noindent \textbf{Problem 2.4.4}
\noindent \textbf{(a)}

\[
\phi'' + \lambda \phi  = 0, \quad \phi(0) = 0, \quad \phi'(L) = 0
\]

\noindent Case 1: $\lambda < 0$. Let $\lambda = -k^2$ for $k>0$.
\[
\phi(x) = A\cosh(kx) + B\sinh(kx)
\]
\[
\phi(0) = A = 0 \implies \phi(x) = B\sinh(kx)
\]
\[
\phi'(x) = Bk\cosh(kx) \implies \phi'(L) = Bk\cosh(kL) = 0
\]
Since $k,L>0$, $\cosh(kL) \neq 0$, so $B=0$. This gives the trivial solution.

\medskip
\noindent Case 2: $\lambda = 0$.
\[
\phi(x) = Ax+B
\]
\[
\phi(0) = B = 0 \implies \phi(x) = Ax
\]
\[
\phi'(x) = A \implies \phi'(L) = A = 0
\]
This gives the trivial solution.

\medskip
\noindent Case 3: $\lambda > 0$. Let $\lambda = k^2$ for $k>0$.
\[
\phi(x) = A\cos(kx) + B\sin(kx)
\]
\[
\phi(0) = A = 0 \implies \phi(x) = B\sin(kx)
\]
\[
\phi'(x) = Bk\cos(kx) \implies \phi'(L) = Bk\cos(kL) = 0
\]
For a non-trivial solution, $B \neq 0$, so $\cos(kL) = 0$.
\[
kL = \frac{(2n-1)\pi}{2}, \quad n=1, 2, 3, \dots
\]
The eigenvalues are:
\[
\lambda_n = k_n^2 = \left(\frac{(2n-1)\pi}{2L}\right)^2, \quad n=1, 2, 3, \dots
\]
The corresponding eigenfunctions are:
\[
\phi_n(x) = \sin\left(\frac{(2n-1)\pi x}{2L}\right)
\]

\noindent \textbf{(b)} 

\noindent Case 1: $n=m$.
\begin{align*}
I &= \int_{0}^{L} \sin^2\left(\frac{(2n-1)\pi x}{2L}\right)dx\\
&= \int_{0}^{L} \frac{1}{2}\left(1 - \cos\left(\frac{(2n-1)\pi x}{L}\right)\right) dx\\
&= \frac{1}{2}\left[x - \frac{L}{(2n-1)\pi}\sin\left(\frac{(2n-1)\pi x}{L}\right)\right]_{0}^{L} \\
&= \frac{1}{2}\left(L - \frac{L}{(2n-1)\pi}\sin((2n-1)\pi)\right) - 0 \\
&= \frac{L}{2}
\end{align*}
since $\sin(k\pi)=0$ for any integer $k$.

\noindent Case 2: $n \neq m$.
\begin{align*}
I &= \int_{0}^{L} \sin\left(\frac{(2n-1)\pi x}{2L}\right)\sin\left(\frac{(2m-1)\pi x}{2L}\right)dx\\
&= \frac{1}{2}\int_{0}^{L} \left[ \cos\left(\frac{(n-m)\pi x}{L}\right) - \cos\left(\frac{(n+m-1)\pi x}{L}\right) \right] dx\\
&= \frac{1}{2}\left[\frac{L}{(n-m)\pi}\sin\left(\frac{(n-m)\pi x}{L}\right) - \frac{L}{(n+m-1)\pi}\sin\left(\frac{(n+m-1)\pi x}{L}\right)  \right]_{0}^{L} \\
&= \frac{1}{2} \left( \frac{L}{(n-m)\pi}\sin((n-m)\pi) - \frac{L}{(n+m-1)\pi}\sin((n+m-1)\pi) \right) - 0 \\
&= 0
\end{align*}


\end{document}
