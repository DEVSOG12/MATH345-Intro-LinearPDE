\documentclass[8pt,landscape,twoside]{extarticle}
\usepackage[margin=0.15in]{geometry}
\usepackage{amsmath,amssymb,mathtools}
\usepackage{multicol}
\usepackage{physics}
\usepackage{enumitem}
\usepackage{titlesec}

% Compact layout settings
\setlength{\columnsep}{0.15in}
\setlength{\parskip}{0pt}
\setlength{\parindent}{0pt}
\pagestyle{empty}

% Remove list spacing
\setlist{nosep,leftmargin=*}

% Reduce section heading spacing
\titlespacing*{\section}{0pt}{1pt}{0.5pt}
\titlespacing*{\subsection}{0pt}{1pt}{0.5pt}
\titleformat{\section}{\bfseries\small}{}{0em}{}
\titleformat{\subsection}{\bfseries\footnotesize}{}{0em}{}

% Reduce spacing around equations
\AtBeginDocument{
  \setlength{\abovedisplayskip}{1.5pt}
  \setlength{\belowdisplayskip}{1.5pt}
  \setlength{\abovedisplayshortskip}{0pt}
  \setlength{\belowdisplayshortskip}{0pt}
}


\begin{document}
\begin{multicols}{2}
\section*{Mr. Sturm Lious Section}

% \subsection*{1.1 General form}

\textbf{General form}. A (regular) Sturm--Liouville eigenvalue problem on $(a,b)$ has the form
\begin{equation}
  \label{SL-general}
  \frac{d}{dx}\!\Big(p(x)\,\phi'(x)\Big) + \big(\lambda r(x) + q(x)\big)\,\phi(x) = 0,
\end{equation}
with boundary conditions of the form
\begin{align*}
  \alpha_1 \phi(a) + \alpha_2 \phi'(a) &= 0,\\
  \beta_1  \phi(b) + \beta_2  \phi'(b) &= 0,
\end{align*}
(where $(\alpha_1,\alpha_2)$ and $(\beta_1,\beta_2)$ are not both zero).

Typical choices:
\begin{itemize}
  \item Dirichlet: $\phi(a)=0$, $\phi(b)=0$.
  \item Neumann: $\phi'(a)=0$, $\phi'(b)=0$.
  \item Robin: $\phi(a)-h_1\phi'(a)=0$, $\phi(b)-h_2\phi'(b)=0$, etc.
\end{itemize}

Assume
\[
p(x) > 0,\qquad r(x) > 0 \quad\text{on }(a,b).
\]

\subsection*{1.2 Weight, inner product, orthogonality}

The \emph{weight function} is
\[
w(x) = r(x).
\]

Define the inner product
\[
\langle f,g\rangle = \int_a^b w(x)\,f(x)\,g(x)\,dx
= \int_a^b r(x)f(x)g(x)\,dx.
\]

If $\phi_m,\phi_n$ are eigenfunctions corresponding to distinct eigenvalues
$\lambda_m\neq\lambda_n$ of \eqref{SL-general}, then
\[
\boxed{\int_a^b r(x)\,\phi_m(x)\,\phi_n(x)\,dx = 0.}
\]
That is, eigenfunctions for different eigenvalues are orthogonal with respect
to the weight $r$.

\subsection*{1.3 Rayleigh quotient (for sign of eigenvalues)}

Rewriting \eqref{SL-general} as
\[
-(p\phi')' - q\phi = \lambda r\phi,
\]
multiplying by $\phi$ and integrating, then integrating by parts,
gives for any nontrivial eigenfunction $\phi$:
\[
\lambda
=
\frac{\displaystyle\int_a^b\bigl[\,p(x)\,(\phi'(x))^2 - q(x)\,\phi(x)^2\,\bigr]\,dx}
     {\displaystyle\int_a^b r(x)\,\phi(x)^2\,dx}.
\]

Typical sign arguments:
\begin{itemize}
  \item If $p>0$, $r>0$, $q\le0$, then $\lambda\ge0$.
  \item If $q<0$ somewhere (and not identically zero), often $\lambda>0$.
\end{itemize}

%%%%%%%%%%%%%%%%%%%%%%%%%%%%%%%%%%%%%%%%%%%%%%%%%%%%%%%%%%%%
% PDE TEMPLATES: HEAT & WAVE WITH VARIABLE COEFFICIENTS
%%%%%%%%%%%%%%%%%%%%%%%%%%%%%%%%%%%%%%%%%%%%%%%%%%%%%%%%%%%%

\section*{2. Separation of variables: general templates}

\subsection*{2.1 Heat-type equations with variable coefficients}

General 1D heat equation:
\begin{equation}
  \label{heat-var}
  r(x)\,u_t = \frac{\partial}{\partial x}\Big(p(x)\,u_x\Big) + q(x)\,u,
  \qquad a<x<b,\ t>0.
\end{equation}
Here
\[
r(x)>0,\quad p(x)>0.
\]

Assume homogeneous boundary conditions that make the operator
self-adjoint (e.g.\ Dirichlet/Neumann/Robin of SL type).

\textbf{Separation of variables.} Try $u(x,t)=\phi(x)T(t)$:
\[
r\phi T' = (p\phi')'T + q\phi T.
\]
Divide by $r\phi T$:
\[
\frac{T'}{T}
=
\frac{(p\phi')'+q\phi}{r\phi}
= -\lambda.
\]
Hence
\[
T'(t) + \lambda T(t) = 0
\quad\Rightarrow\quad
T(t) = e^{-\lambda t},
\]
and the spatial SL problem is
\begin{equation}
  \label{heat-eig}
  (p\phi')' + q\phi + \lambda r\phi = 0
  \qquad\Longleftrightarrow\qquad
  -(p\phi')' - q\phi = \lambda r\phi.
\end{equation}

Let $\{\lambda_n,\phi_n(x)\}_{n\ge1}$ be the eigenpairs of \eqref{heat-eig}
satisfying the BCs, with eigenfunctions orthogonal in the weight $r(x)$.

\textbf{General solution.}
Expand the initial data $u(x,0)=f(x)$ in the eigenfunctions:
\[
f(x) = \sum_{n=1}^\infty a_n\phi_n(x),
\]
with coefficients
\begin{equation}
  \label{coeff-general-heat}
  a_n
  =
  \frac{\displaystyle\int_a^b r(x)\,f(x)\,\phi_n(x)\,dx}
       {\displaystyle\int_a^b r(x)\,\phi_n(x)^2\,dx}.
\end{equation}
Then the solution is
\begin{equation}
  \label{heat-solution}
  u(x,t)
  = \sum_{n=1}^\infty a_n e^{-\lambda_n t}\,\phi_n(x).
\end{equation}

\textbf{Long-time behavior.}
If all $\lambda_n>0$, then $u(x,t)\to0$ as $t\to\infty$.
If there is a zero eigenvalue $\lambda_0=0$ with eigenfunction $\phi_0$,
then
\[
u(x,t) = a_0\phi_0(x) + \sum_{n\ge1} a_n e^{-\lambda_n t}\phi_n(x)
\]
and
\[
\lim_{t\to\infty}u(x,t) = a_0\phi_0(x).
\]

Example: For Neumann (insulated) heat problems one typically has
$\phi_0(x)=1$ and
\[
a_0 = \frac{\displaystyle\int_a^b r(x)f(x)\,dx}
            {\displaystyle\int_a^b r(x)\,dx},
\]
so the solution tends to the weighted spatial average of the initial data.

\subsection*{2.2 Wave-type equations with variable coefficients}

General 1D wave equation:
\begin{equation}
  \label{wave-var}
  r(x)\,u_{tt} = \frac{\partial}{\partial x}\Big(p(x)\,u_x\Big) + q(x)\,u,
  \qquad a<x<b,\ t>0.
\end{equation}

Try $u(x,t)=\phi(x)G(t)$:
\[
r\phi G'' = (p\phi')'G + q\phi G.
\]
Divide by $r\phi G$:
\[
\frac{G''}{G}
=
\frac{(p\phi')'+q\phi}{r\phi}
= -\lambda.
\]

Time ODE:
\[
G'' + \lambda G = 0
\quad\Rightarrow\quad
G(t) = A\cos(\sqrt{\lambda}\,t) + B\sin(\sqrt{\lambda}\,t).
\]

Spatial eigenvalue problem (same SL operator as heat-type case):
\begin{equation}
  \label{wave-eig}
  (p\phi')' + q\phi + \lambda r\phi = 0.
\end{equation}

Let $\{\lambda_n,\phi_n\}_{n\ge1}$ be eigenpairs. The general solution is
\begin{equation}
  \label{wave-expansion}
  u(x,t)=\sum_{n=1}^\infty
  \bigl(A_n\cos(\sqrt{\lambda_n}\,t)+B_n\sin(\sqrt{\lambda_n}\,t)\bigr)\,\phi_n(x).
\end{equation}

Given initial conditions
\[
u(x,0)=f(x),\qquad u_t(x,0)=g(x),
\]
we obtain
\[
f(x) = \sum_{n=1}^\infty A_n \phi_n(x),\qquad
g(x) = \sum_{n=1}^\infty B_n\sqrt{\lambda_n}\,\phi_n(x).
\]

Using orthogonality in weight $r(x)$:
\begin{align}
  \label{An-general-wave}
  A_n &=
  \frac{\displaystyle\int_a^b r(x)\,f(x)\,\phi_n(x)\,dx}
       {\displaystyle\int_a^b r(x)\,\phi_n(x)^2\,dx},\\
  \label{Bn-general-wave}
  B_n &=
  \frac{\displaystyle\int_a^b r(x)\,g(x)\,\phi_n(x)\,dx}
       {\displaystyle\sqrt{\lambda_n}\int_a^b r(x)\,\phi_n(x)^2\,dx}.
\end{align}

Special case (as in 5.5.16):
\begin{itemize}
  \item For $u_{tt}=c^2(x)u_{xx}$ with $u(0,t)=u(L,t)=0$ we have
  \[
  p(x)=1,\quad q(x)=0,\quad r(x)=\frac{1}{c^2(x)}.
  \]
  Orthogonality is with weight $1/c^2(x)$.
\end{itemize}

%%%%%%%%%%%%%%%%%%%%%%%%%%%%%%%%%%%%%%%%%%%%%%%%%%%%%%%%%%%%
% % COMMON SPECIAL CASES
% %%%%%%%%%%%%%%%%%%%%%%%%%%%%%%%%%%%%%%%%%%%%%%%%%%%%%%%%%%%%

% \section*{3. Common special cases}

% \subsection*{3.1 Constant-coefficient heat equation (Dirichlet)}

% \[
% u_t = k u_{xx} - \beta u, \quad 0<x<L,\ t>0,
% \]
% \[
% u(0,t)=u(L,t)=0,\quad u(x,0)=f(x).
% \]

% Eigenproblem:
% \[
% \phi'' + \mu^2\phi = 0,\quad \phi(0)=\phi(L)=0
% \quad\Rightarrow\quad
% \mu_n=\frac{n\pi}{L},\ \phi_n(x)=\sin\frac{n\pi x}{L}.
% \]

% Separation gives decay rates $k\mu_n^2+\beta$,
% so the solution is
% \[
% u(x,t)
% =
% \sum_{n=1}^\infty a_n
% \exp\!\Big[-\big(k(n\pi/L)^2+\beta\big)t\Big]\,
% \sin\frac{n\pi x}{L},
% \]
% with
% \[
% a_n
% =
% \frac{2}{L}\int_0^L f(x)\,\sin\frac{n\pi x}{L}\,dx.
% \]

% \subsection*{3.2 Constant-coefficient heat equation (Neumann)}

% \[
% u_t = k u_{xx},\quad 0<x<L,
% \]
% \[
% u_x(0,t)=u_x(L,t)=0,\quad u(x,0)=f(x).
% \]

% Eigenproblem:
% \[
% \phi'' + \mu^2\phi = 0,\quad \phi'(0)=\phi'(L)=0
% \]
% gives
% \[
% \phi_0(x)=1,\ \lambda_0=0;\qquad
% \phi_n(x)=\cos\frac{n\pi x}{L},\ \lambda_n=k(n\pi/L)^2,\ n\ge1.
% \]

% Solution:
% \[
% u(x,t)
% = a_0 + \sum_{n=1}^\infty a_n
% e^{-k(n\pi/L)^2 t}\cos\frac{n\pi x}{L}.
% \]

% Coefficients (weight $r(x)=1$):
% \[
% a_0 = \frac{1}{L}\int_0^L f(x)\,dx,\qquad
% a_n = \frac{2}{L}\int_0^L f(x)\,\cos\frac{n\pi x}{L}\,dx.
% \]

% Long-time limit:
% \[
% \lim_{t\to\infty}u(x,t)=a_0 \quad\text{(spatial average of $f$)}.
% \]

% \subsection*{3.3 Constant-coefficient wave equation (Dirichlet)}

% \[
% u_{tt}=c^2 u_{xx},\quad 0<x<L,
% \]
% \[
% u(0,t)=u(L,t)=0,\quad u(x,0)=f(x),\quad u_t(x,0)=g(x).
% \]

% Eigenfunctions: $\phi_n(x)=\sin\frac{n\pi x}{L}$, eigenvalues
% $\lambda_n=(n\pi/L)^2$.

% Solution:
% \[
% u(x,t)
% =
% \sum_{n=1}^\infty
% \Big(A_n\cos(c\sqrt{\lambda_n}t)+B_n\sin(c\sqrt{\lambda_n}t)\Big)
% \sin\frac{n\pi x}{L},
% \]
% with
% \begin{align*}
% A_n &= \frac{2}{L}\int_0^L f(x)\,\sin\frac{n\pi x}{L}\,dx,\\
% B_n &= \frac{2}{c n\pi}\int_0^L g(x)\,\sin\frac{n\pi x}{L}\,dx.
% \end{align*}

%%%%%%%%%%%%%%%%%%%%%%%%%%%%%%%%%%%%%%%%%%%%%%%%%%%%%%%%%%%%
% EULER–CAUCHY (EQUIDIMENSIONAL) ODES
%%%%%%%%%%%%%%%%%%%%%%%%%%%%%%%%%%%%%%%%%%%%%%%%%%%%%%%%%%%%

\section*{4. Equidimensional (Euler--Cauchy) ODEs}

A typical equidimensional ODE:
\begin{equation}
  \label{EC-general}
  x^2\phi'' + a x\phi' + b(x)\phi = 0,
  \qquad x>0,
\end{equation}
where $b(x)$ may be constant or contain eigenvalue parameters.

\subsection*{4.1 Power-law ansatz}

Try $\phi(x)=x^m$:
\[
\phi' = m x^{m-1},\quad
\phi'' = m(m-1)x^{m-2}.
\]

Substitute into \eqref{EC-general} (for constant $b$) to get the algebraic
(indicial) equation:
\[
m(m-1) + a m + b = 0.
\]

Solve for $m$:
\begin{itemize}
  \item Two distinct real roots $m_1\neq m_2$:
  \[
  \phi(x)=C_1x^{m_1}+C_2x^{m_2}.
  \]
  \item Repeated root $m_1=m_2=m$:
  \[
  \phi(x) = C_1 x^m + C_2 x^m\ln x.
  \]
  \item Complex roots $m=\alpha\pm i\beta$:
  \[
  \phi(x) = x^{\alpha}\Big(C_1\cos(\beta\ln x)+C_2\sin(\beta\ln x)\Big).
  \]
\end{itemize}

Example pattern from the practice problems:
\[
x^2\phi'' + 2x\phi' + (\lambda -1)\phi = 0
\quad\Rightarrow\quad
m^2+m+(\lambda-1)=0.
\]
If $\lambda>5/4$ the roots are complex and the general real solution is
\[
\phi(x) = x^{-1/2}\Big(A\cos(\mu\ln x) + B\sin(\mu\ln x)\Big),
\]
with $\mu = \tfrac12\sqrt{4\lambda-5}$, etc.

%%%%%%%%%%%%%%%%%%%%%%%%%%%%%%%%%%%%%%%%%%%%%%%%%%%%%%%%%%%%
% ORTHOGONALITY PROOF PATTERN
%%%%%%%%%%%%%%%%%%%%%%%%%%%%%%%%%%%%%%%%%%%%%%%%%%%%%%%%%%%%

\section*{5. Orthogonality proof pattern}

Given an SL problem
\[
(p\phi')' + (\lambda r+q)\phi=0
\]
with appropriate self-adjoint BCs, the orthogonality for $\lambda_m\neq\lambda_n$
is obtained by:

\begin{enumerate}
  \item Write the equations for $\phi_m$ and $\phi_n$:
  \[
  (p\phi_m')' + (\lambda_m r+q)\phi_m = 0,\qquad
  (p\phi_n')' + (\lambda_n r+q)\phi_n = 0.
  \]
  \item Multiply the first by $\phi_n$, the second by $\phi_m$, and subtract:
  \[
  \phi_n(p\phi_m')' - \phi_m(p\phi_n')'
  + (\lambda_m-\lambda_n)r\phi_m\phi_n = 0.
  \]
  \item Recognize a derivative:
  \[
  \phi_n(p\phi_m')' - \phi_m(p\phi_n')'
  = \frac{d}{dx}\big(p(\phi_n\phi_m' - \phi_m\phi_n')\big).
  \]
  \item Integrate over $(a,b)$:
  \[
  \big[p(\phi_n\phi_m' - \phi_m\phi_n')\big]_a^b
  + (\lambda_m-\lambda_n)\int_a^b r\phi_m\phi_n\,dx = 0.
  \]
  \item Use the boundary conditions to show the boundary term is $0$; conclude
  \[
  (\lambda_m-\lambda_n)\int_a^b r(x)\phi_m(x)\phi_n(x)\,dx = 0,
  \]
  hence orthogonality for $\lambda_m\neq\lambda_n$.
\end{enumerate}

\noindent \textbf{Diri}
\[
u(x,t) = \sum_{n=1}^\infty
\Big[ A_n \cos(\omega_n t) + B_n \sin(\omega_n t) \Big]
\sin\!\frac{n\pi x}{L},\qquad \omega_n = \frac{c n\pi}{L}.
\]

% Coefficients from initial data:
\[
A_n = \frac{2}{L}\int_0^L f(x)\,\sin\frac{n\pi x}{L}\,dx,
\]
\[
B_n = \frac{2}{c n\pi}\int_0^L g(x)\,\sin\frac{n\pi x}{L}\,dx.
\]

\noindent \textbf{Neu}
\[
u(x,t) = A_0 + B_0 t
+ \sum_{n=1}^\infty
\Big[ A_n \cos(\omega_n t) + B_n \sin(\omega_n t) \Big]
\cos\!\frac{n\pi x}{L},\qquad \omega_n = \frac{c n\pi}{L}.
\]

% Coefficients from initial data:

% zero mode:
\[
A_0 = \frac{1}{L}\int_0^L f(x)\,dx, \qquad
B_0 = \frac{1}{L}\int_0^L g(x)\,dx.
\]

% higher modes (n>=1):
\[
A_n = \frac{2}{L}\int_0^L f(x)\,\cos\frac{n\pi x}{L}\,dx,
\]
\[
B_n = \frac{2}{c n\pi}\int_0^L g(x)\,\cos\frac{n\pi x}{L}\,dx,
\qquad n\ge1.
\]

\noindent \textbf{Mixed BC: fixed at x=0, free at x=L}
\[
\phi_n(x) = \sin\!\left(\frac{(2n+1)\pi x}{2L}\right), \qquad
\lambda_n = \left(\frac{(2n+1)\pi}{2L}\right)^2,
\qquad \omega_n = c \sqrt{\lambda_n}
                = \frac{(2n+1)\pi c}{2L}.
\]

% General solution:
\[
u(x,t) = \sum_{n=0}^\infty
\Big[ A_n \cos(\omega_n t) + B_n \sin(\omega_n t) \Big]
\sin\!\left(\frac{(2n+1)\pi x}{2L}\right).
\]

% Coefficients (using orthogonality: ∫_0^L phi_n^2 dx = L/2):
\[
A_n = \frac{2}{L}\int_0^L f(x)
      \sin\!\left(\frac{(2n+1)\pi x}{2L}\right)\,dx,
\]
\[
B_n = \frac{2}{L\omega_n}\int_0^L g(x)
      \sin\!\left(\frac{(2n+1)\pi x}{2L}\right)\,dx
    = \frac{4}{c(2n+1)\pi}
      \int_0^L g(x)\sin\!\left(\frac{(2n+1)\pi x}{2L}\right)\,dx.
\]

\noindent \textbf{Mixed BC: free at x=0, fixed at x=L}
\[
\phi_n(x) = \cos\!\left(\frac{(2n+1)\pi x}{2L}\right), \qquad
\lambda_n = \left(\frac{(2n+1)\pi}{2L}\right)^2,
\qquad \omega_n = \frac{(2n+1)\pi c}{2L}.
\]

% General solution:
\[
u(x,t) = \sum_{n=0}^\infty
\Big[ A_n \cos(\omega_n t) + B_n \sin(\omega_n t) \Big]
\cos\!\left(\frac{(2n+1)\pi x}{2L}\right).
\]

% Coefficients (again, ∫_0^L phi_n^2 dx = L/2):
\[
A_n = \frac{2}{L}\int_0^L f(x)
      \cos\!\left(\frac{(2n+1)\pi x}{2L}\right)\,dx,
\]
\[
B_n = \frac{2}{L\omega_n}\int_0^L g(x)
      \cos\!\left(\frac{(2n+1)\pi x}{2L}\right)\,dx
    = \frac{4}{c(2n+1)\pi}
      \int_0^L g(x)\cos\!\left(\frac{(2n+1)\pi x}{2L}\right)\,dx.
\]

%%%%%%%%%%%%%%%%%%%%%%%%%%%%%%%%%%%%%%%%%%%%%%%%%%%%%%%%%%%%%%%%%%%%%%
% Fourier Series: Even/Odd Extensions, Jumps, and S–L Summary
%%%%%%%%%%%%%%%%%%%%%%%%%%%%%%%%%%%%%%%%%%%%%%%%%%%%%%%%%%%%%%%%%%%%%%

\section*{1. Full Fourier Series on \([-L,L]\)}

A \(2L\)-periodic function \(F(x)\) has Fourier series
\[
F(x) \sim \frac{a_0}{2} + \sum_{n=1}^\infty
\left( a_n \cos\frac{n\pi x}{L} + b_n \sin\frac{n\pi x}{L} \right),
\]
with
\[
a_0 = \frac{1}{L} \int_{-L}^{L} F(x)\,dx,\quad
a_n = \frac{1}{L} \int_{-L}^{L} F(x)\cos\frac{n\pi x}{L}\,dx,\quad
b_n = \frac{1}{L} \int_{-L}^{L} F(x)\sin\frac{n\pi x}{L}\,dx.
\]

If \(F\) is piecewise \(C^1\) (piecewise smooth, finitely many jumps) and
\(2L\)-periodic, then:

\[
\lim_{N\to\infty} S_N(x) =
\begin{cases}
F(x), & F\ \text{continuous at }x,\\[2mm]
\displaystyle
\frac{1}{2}\big(F(x^-)+F(x^+)\big),
& x \text{ a jump point},
\end{cases}
\]
where \(S_N(x)\) is the \(N\)-th partial sum of the Fourier series.

%%%%%%%%%%%%%%%%%%%%%%%%%%%%%%%%%%%%%%%%%%%%%%%%%%%%%%%%%%%%%%%%%%%%%%
% \section*{2. Half-Range Expansions on \([0,L]\)}

% We start with a function defined only on \(0\le x\le L\), say \(f(x)\).

% \subsection*{2.1 Even Extension (Cosine Series)}

% Define the \emph{even extension}
% \[
% f_e(x) =
% \begin{cases}
% f(x),   & 0\le x\le L,\\[1mm]
% f(-x),  & -L\le x<0,
% \end{cases}
% \]
% and then extend to all \(x\in\mathbb{R}\) by periodicity
% \(f_e(x+2L)=f_e(x)\).

% Then \(f_e(-x)=f_e(x)\) and its Fourier series has only cosines.
% The \textbf{Fourier cosine series} of \(f\) on \([0,L]\) is
% \[
% f(x) \sim \frac{a_0}{2} + \sum_{n=1}^\infty a_n \cos\frac{n\pi x}{L},
% \quad 0\le x\le L,
% \]
% with
% \[
% a_0 = \frac{2}{L} \int_0^{L} f(x)\,dx,\qquad
% a_n = \frac{2}{L} \int_0^{L} f(x)\cos\frac{n\pi x}{L}\,dx.
% \]

% \subsection*{2.2 Odd Extension (Sine Series)}

% Define the \emph{odd extension}
% \[
% f_o(x) =
% \begin{cases}
% f(x),   & 0\le x\le L,\\[1mm]
% -\,f(-x), & -L\le x<0,
% \end{cases}
% \]
% and again extend by periodicity \(f_o(x+2L)=f_o(x)\).

% Then \(f_o(-x)=-f_o(x)\) and its Fourier series has only sines.
% The \textbf{Fourier sine series} of \(f\) on \([0,L]\) is
% \[
% f(x) \sim \sum_{n=1}^\infty b_n \sin\frac{n\pi x}{L},
% \quad 0\le x\le L,
% \]
% with
% \[
% b_n = \frac{2}{L} \int_0^{L} f(x)\sin\frac{n\pi x}{L}\,dx.
% \]

% \subsection*{2.3 Where Do Jump Discontinuities Come From?}

% After you form \(f_e\) or \(f_o\) on \([-L,L]\), you make it periodic.
% Possible jumps occur at the points where periods meet:
% \[
% x = kL \quad (k\in\mathbb{Z}).
% \]
% To check for a jump at a point \(x_0\), compare the one-sided limits:
% \[
% \text{if }F(x_0^-)\neq F(x_0^+)\ \Rightarrow\ \text{jump at }x_0;
% \quad\text{otherwise, no jump.}
% \]
% At a jump, the Fourier series converges to
% \(\tfrac{1}{2}\big(F(x_0^-)+F(x_0^+)\big)\).

%%%%%%%%%%%%%%%%%%%%%%%%%%%%%%%%%%%%%%%%%%%%%%%%%%%%%%%%%%%%%%%%%%%%%%
% \section*{3. Example: \(f(x)=1+3x\) on \([0,5]\) (\(L=5\))}

% We consider the interval \([-L,L]=[-5,5]\) and period \(2L=10\).

% \subsection*{3.1 Odd Extension (Sine Series)}

% On \(0\le x\le 5\): \(f(x)=1+3x\).

% On \(-5\le x<0\): define
% \[
% f_o(x) = -f(-x) = -\big(1+3(-x)\big) = -1+3x.
% \]

% Thus
% \[
% f_o(0^+) = f(0)=1,\qquad f_o(0^-) = -f(0)=-1,
% \]
% so there is a jump of size \(2\) at \(x=0\).
% We then extend periodically with period \(10\), so the same jump
% behavior occurs at \(x = 10k\) for all integers \(k\).

% The Fourier sine series of \(f\) on \([0,5]\) approximates this
% odd, piecewise linear, \(10\)-periodic function.

% \subsection*{3.2 Even Extension (Cosine Series)}

% On \(-5\le x<0\): define
% \[
% f_e(x) = f(-x) = 1+3(-x) = 1-3x.
% \]

% Here
% \[
% f_e(0^-)=f(0)=1=f_e(0^+),
% \]
% so no jump at \(x=0\). At \(x=\pm 5\) the left/right copies also match,
% so \(f_e\) is continuous and even: \(f_e(x)=1+3|x|\) on \([-5,5]\),
% extended periodically with period \(10\).

% The Fourier cosine series of \(f\) on \([0,5]\) approximates this
% even, piecewise linear, \(10\)-periodic function.

%%%%%%%%%%%%%%%%%%%%%%%%%%%%%%%%%%%%%%%%%%%%%%%%%%%%%%%%%%%%%%%%%%%%%%
\section*{4. Sturm--Liouville Summary (Key Properties)}

Consider a regular Sturm--Liouville problem
\[
\frac{d}{dx}\big(p(x)y'(x)\big) + \big(\lambda\,w(x) - q(x)\big)y(x) = 0,
\]
on \([a,b]\), with \(p>0\), \(w>0\), and appropriate self-adjoint
(separated) boundary conditions. Then:

\begin{enumerate}
  \item \textbf{Eigenvalues are real.}\\[1mm]
  All eigenvalues \(\lambda_n\) are real numbers.

  \item \textbf{Eigenvalues are discrete and ordered.}\\[1mm]
  There is an infinite sequence
  \[
  \lambda_1 < \lambda_2 < \lambda_3 < \dots,\qquad
  \lambda_n \to +\infty.
  \]

  \item \textbf{Orthogonality w.r.t. the weight.}\\[1mm]
  If \(y_m,y_n\) correspond to different eigenvalues
  \(\lambda_m\neq\lambda_n\), then
  \[
  \int_a^b w(x)\,y_m(x)\,y_n(x)\,dx = 0.
  \]

  \item \textbf{One-dimensional eigenspaces.}\\[1mm]
  For each \(\lambda_n\), all eigenfunctions are multiples of a single
  eigenfunction (no independent ``extra'' eigenfunctions for the same
  eigenvalue under separated BCs).

  \item \textbf{Zero (oscillation) count.}\\[1mm]
  The \(n\)-th eigenfunction \(y_n\) has exactly \(n-1\) zeros in the
  open interval \((a,b)\).
\end{enumerate}

(Additionally, in applications one often uses that the eigenfunctions
\(\{y_n\}\) form a complete orthogonal set in \(L^2_w[a,b]\), so any
sufficiently nice function can be expanded in a Fourier series of
eigenfunctions.)


\end{multicols}
\end{document}